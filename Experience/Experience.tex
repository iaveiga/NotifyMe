\documentclass[a4paper,11pt]{article}
\usepackage[utf8]{inputenc}
\usepackage{graphicx}

\begin{document}

\author{
	Aveiga Iván \ Enriquez Wilson \ Sornoza Andrés
}
\title{Notify Me: Observaciones, Conclusiones, Experiencias }
\maketitle
\begin{figure}[h]
\centering
\includegraphics[width=0.7\linewidth]{./logo}
\end{figure}

\newpage
\section{Observaciones}
Como observaciones podemos destacar el trabajo de usar la API de GoogleMaps en aplicaciones móviles y el uso de desarrollo móvil en Android. También el uso de software de versionamiento para controlar las versiones del proyecto. \\

También podemos observar el proceso de desarrollo de una aplicación móvil, lo cuál incluye desde el proceso de diseño de la aplicación, y concluyendo en el testeo de la misma en un dispotivio móvil.

\section{Conclusiones}
Podemos concluir que el uso de software de versionamiento es muy útil para el desarrollo del proyecto, teniendo ventajas que la forma de trabajo tradicional no brinda. \\

Fue beneficioso trabajar con el API de Google Maps, ya que la experiencia que tiene Google en el asunto de mapas es casi imposible de igualar por otros servicios de Mapas, sean estos gratuitos o de pago. \\

Se analizó la idea de trabajar con Google Places para el uso de categorías, esto se analizaría en futuras versiones de NotifyMe.

\section{Experiencia}
El equipo de NotifyMe considera la experiencia positiva, y muy beneficiosa desde el punto de vista académico, ya que el uso de herramientas de versionamiento es muy útil como experiencia, es algo que se usará para futuros proyectos no solo a nivel académico. \\

Otro beneficio fue trabajar con tecnologías móviles, ya que esto permite adquirir poco a poco experiencia en esta área ya que hoy en día representa un gran mercado competitivo.




\end{document}